\documentclass{beamer}
\usepackage[spanish]{babel}
\usepackage[utf8]{inputenc}
\usepackage{graphicx}

\usetheme{Madrid}
\definecolor{MiVioleta}{RGB}{122,59,122}
\definecolor{MiAzul}{RGB}{0,88,147}
\definecolor{MiGris}{RGB}{56,61,66}
\setbeamercolor{palette primary}{use=structure, fg=white, bg=MiVioleta}
\setbeamercolor{palette secundary}{use=structure, fg=white, bg=MiAzul}
\setbeamercolor{palette tertiary}{use=structure, fg=white, bg=MiGris}

\title[Beamer]{El numero $\pi$}
\author[Técnicas Experimentales]{Andrés Francisco Palenzuela}
\institute{Fac. Mat.}
\date[24-04-2014]{24 de Abril de 2014}

\begin{document}

\begin{frame}

  \includegraphics[width=0.15\textwidth]{img/ullesc.eps}
  \hspace*{7.5cm}
  \includegraphics[width=0.16\textwidth]{img/fmatesc.eps}
\titlepage

  \begin{scriptsize}
    \begin{center}
     Facultad de Matemáticas \\
     Universidad de La Laguna
    \end{center}
  \end{scriptsize}

\end{frame} 


\begin{frame}
  \frametitle{Índice}  
  \tableofcontents[pausesections]
\end{frame}  


\section{Primera Sección}

\begin{frame}

\frametitle{Que es $\pi$}

El número $\pi$, equivale a la constante que relaciona el perímetro o longitud de una circunferencia con su diámetro. Se trata de un valor con un infinito 
número de decimales, cuya secuencia comienza de la siguiente manera: 3,1415926535897932384626433832795028841….\par
Redondeado en 3,1416, $\pi$ es un número irracional, frecuentemente utilizado en las matemáticas y en la física, además de en otras disciplinas como la 
geometría y la trigonometría.

\end{frame}

\section{Segunda Sección}

\begin{frame}

\frametitle{Ejemplos}
Algunas de las ecuaciones que usan el número $\pi$ son:

\begin{block}{Ejemplo}
  \begin{itemize}
  \item
  Area del cilindro: $2\pi r(r+h)$
  \pause

  \item
  Volumen de una esfera de radio r: $V=(4/3)\pi r^3$
  \pause

  \item
   Longitud de una circunferencia: $C=2\pi r$
  \pause

  \item
  Area del círculo: $A=\pi r^2$
  \pause

  \item
   Area de la esfera: $4\pi r^2$
  \pause

  \end{itemize}
\end{block}

\end{frame}

\section{Bibliografía}

\begin{frame}
  \frametitle{Bibliografía}

  \begin{thebibliography}{10}

    \beamertemplatebookbibitems
    \bibitem[Práctica 11]{practica}  
     Práctica 11, técnicas experimentales. 
    (2014)
    {\small $https://www.google.es/q=wikipedia$} 

    \beamertemplatebookbibitems
    \bibitem[Comandos LaTeX - Página - Fórmulas - Bibliografía]{comandos}  
    Comandos LaTeX - Página - Fórmulas - Bibliografía 
    (2013) 
    {\small $http://campusvirtual.ull.es/1213m2/pluginfile.php/224421/mod_resource/content/3/TeoriaLaTeX.2.pdf$}

  \end{thebibliography}
\end{frame}
\end{document}

